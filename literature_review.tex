It is to be noted that each subsection in this chapter should discuss in narrative form each table that is presented.  

\section{Battery Management systems for UAV}

\subsection{Battery Power Management for Portable Devices - Lead-Acid Battery Charger}
This is a book that talks about different techniques of battery power management starting from understanding the chemical characteristics of battery, different battery charger techniques, battery safety, cell-balancing, battery fuel gauging, etc. the researchers will only focus on implementing battery charger technique using solar panels. 

\subsection{G. Sierra, M. Orchard, K. Goebel, C. Kulkarni, Battery Health Management for Small-size Rotary-wing Electric Unmanned Aerial Vehicles: An Efficient Approach for Constrained Computing Platforms,Reliability Engineering and System Safety, 2018}
This article presents a system of battery management that accurately estimates the State of Charge and predicts the End of Discharge time of lithium-polymer batteries (Li-Po). These batteries are used in small-size multirotors that is efficient in low-cost hardware. The researchers presented a simplified battery model that incorporates State of Charge dependence and electric load dependence by using Artificial Evolution to estimate their parameters. They used Lithium Polymer batteries to adapt to the UAV’s high density energy. They stated that the UAVs flight endurance is in direct relationship to the total weight of it. which makes it best to be powered by Li-Po batteries.
\subsection{Mostafa, T., Muharam, A.,Hattori, R. Wireless Battery Charging System for Drones via Capacitive Power Transfer}

The researchers introduced wireless power transmission as an efficient solution for problems that deals with charging with wires and have integrated this idea in charging UAV by developing a wireless battery charging for UAV. The charging system, composed of a full-bridge power inverter, a low pass filter, a directional coupler, a pi-matching circuit, a center-tapped step up transformer, etc., has shown a 50 percent efficiency improvement in battery charging applications and delivered 12 W.

\subsection{Xiaohui Zhang, Li Liu, Yueling Dai, Tianhe Lu, (2018) Experimental investigation on the online fuzzy energy management of hybrid fuel cell/battery power system for UAVs.}

The researchers made a power optimization technique for UAVs using supercapacitors. Supercapacitors have greater capacitance compared to the regular capacitors because it makes use of the electrodes with higher surface area and thinner dielectrics thus allowing greater power densities as compared to conventional capacitors and Li-Po batteries. The power required by the load is provided by the supercapacitor in a static manner which then would lighten the power requirements for the load thus, improving flight time.

\subsection{Li, G. C. (2014). Circuits and methods for controlling battery management systems.}

 This literature serves as a guideline when designing a battery management system whether which kind of circuit will be most applicable for the Solar Powered UAV. A controller is used for battery management along with three terminals: first terminal which is responsible for receiving power, second terminal where the clock pulse signal is connected, and lastly, the communication circuitry which is coupled to the first and the second terminal in order to detect the clock signal and generate a switching signal depending on the clock pulse detected by the communication circuit.
\subsection{A. de Souza Cândido, R. K. Harrop Galvão and T. Yoneyama, (2014).  Control and energy management for quadrotor, UKACC International Conference on Control (CONTROL), Loughborough, 2014, pp. 343-348. }
This research paper emphasized on the energy management problem by using State-of-Charge update, fault prognosis and guidance system. These steps helped the researchers obtain the data needed from the quadrotor which is its voltage measurements and its mission path. The data obtained can be used to determine and solve the energy saving problems in the quadrotor.







\section{Framework designs for Quadrotor/UAV}

\subsection{W. Wei, T, Mark, S, Nicholas, C, Kelly. System Identification and Flight Control of an Unmanned Quadrotor.}

The researchers built the frame by constructing the aluminum plates in the center which contains most of the electrical components of the UAV. Four pieces of square aluminum tubes were used for the motor arms. These tubes were made hollow so that the UAV is lighter. Additionally, they also used bare-airframe dynamics that exhibit unstable and classical hovering characteristics that are perfectly symmetrical between roll and pitch.

\subsection{Da Ronch,  A., Marques P. (2017). Advanced UAV Aerodynamics, Flight Stability and Control - Novel Concepts, Theory and Applications. John Wiley and Sons.}
The book discusses various new developments in the field of aerospace technology, which includes discussions on UAV frame design and aerodynamics. The researchers will require the discussed design principles in order to construct a suitable frame for the quad-rotor aircraft.

\subsection{Craciun, D., Kuantama, E., Tarca, I. C., Tarca, R. C. (2017). New Advances in Mechanisms, Mechanical Transmissions and Robotics: Proceedings of The Joint International Conference of the XII International Conference on Mechanisms and Mechanical Transmissions (MTM) and the XXIII International Conference on Robotics (Robotics ’16).}
 A chapter in this book named Quadcopter Propeller Design and Performance Analysis talks about the design process of a quadcopter uav, namely on the selection of rotor and propellor. The authors of the book mention using simulations to determine the behavior of air flow for different types of propellers to install, depending on the thrust required.

\subsection{S. Jashnani, T.R. Nada, M. Ishfaq, A. Khamker, P. Shaholia, Sizing and preliminary hardware testing of solar powered UAV, The Egyptian Journal of Remote Sensing and Space Science, 2013.} This literature includes guidelines on the estimation of the solar panel area, and weight estimation to be used for UAVs. It also includes guidelines on testing the parameters needed for designing a good quadrotor.

\subsection{A. Ailon and S. Arogeti, "On set-point control of a quadrotor-type helicopter with a suspended load," 2016 2nd International Conference on Control, Automation and Robotics (ICCAR), Hong Kong, 2016, pp. 194-199.}
This research focuses on quadrotors transporting a payload with suspended cables. The equations of motion were used to compute for the body frame size of the quadrotor with respect to the center of gravity of the payload. The combination of open loop and closed loop control strategies ensured that the quadrotor will have stability even when there is a suspended load.

\subsection{X. Hu and X. Huang, "Orthogonal design and optimization of flight stability test for the quadrotor unmanned aerial vehicle," 2017 IEEE International Conference on Unmanned Systems (ICUS), Beijing, 2017, pp. 343-346.}
This paper aims to optimize the flight stability of a quadrotor by adopting an orthogonal design to study the factors that affect flight stability. Wind and throttle control were taken into account as research objects. MATLAB software was used in to simulate disturbances in the flight of the quadrotor. It was found out that orthogonal design is an effective method for testing the flight stability of the drone.

\section{Solar Powered Technologies for Quadrator/UAV}
\subsection{Lachica, H. C., Jr., Laconico, C. D., Pepito, C. R., Rillera, H. T., Sebastian, D. L., Magsino, E. R., and Abad, A. C. (2016).}
Development of a Remote-Controlled Quadrotor with Solar Recharging and Emergency Landing Capabilities (Unpublished doctoral dissertation). De La Salle University, Manila, Philippines. The quadrotor in this study is remote-controlled, has a capability of recharging using solar energy, and emergency landing capabilities. Power switching circuits were used between two battery packs so that when a battery reached a certain voltage level, the second battery will be switched in allowing the first battery to be recharged using solar energy.

\subsection{Shaheed, H. M., et al. (2015). Flying by the Sun only. Aerospace Science and Technology. Hassan Shaheed, et al.}
They created a prototype of a quadcopter which runs only using solar energy. The required output power of their UAV requires 130W to provide sufficient thrust to the propulsion system. Aside from that, the type of PV cells used were monocrystalline cells. A total of 36 cells with a surface area of 15.7cm2 per cell. Thus, generating 136.8W of output power which weights 370.8g. After the design of the UAV the total mass was 925g.

\subsection{T. Kalyani, et.al. (2018). Energy Efficient Sun Synchronous Solar Panel.}
These researchers were able to create a solar panel that is synchronous to the sun where the panels could move according to sunlight. With this, the output energy is 30 percent more compared to those solar panels without a tracking system. To create the system, the researchers used Arduino UNO board-atmega 328, LDR sensors-5mm, servo motor-sg90,5v., and solar panel. With this system, the LDR sensor is able to detect the sun light intensity wherein transmitting the information to the microcontroller for the changing of the axis of the panel resulting to more energy being received by the solar panel.

\subsection{Rajendran, P., and Smith, H. (2018). Development of Design Methodology for a Small Solar-Powered Unmanned Aerial Vehicle. International Journal Of Aerospace Engineering, 1-10.}
The aim of this literature is to improve the shortcomings of present designs of UAVs by proposing three new design aspects, three new design properties, and a new design feature. These three are incorporated to a small solar UAV design model hoping that an improvement of power consumption-to-take-off mass ratio will be achieved through the use of solar energy and battery.



\section{Lacking in the Approaches}
The researchers will be required to determine the optimal voltage threshold for the battery switching operation, as well as the types of batteries to be used. Another key element not found in the sources would involve the implementation of the emergency landing system which would work with the battery-switching operation on determining when the activation sequence would commence.








