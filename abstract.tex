\chapter*{Abstract}

\begin{comment}
Keep your abstract short by giving the gist/nutshell of your \MakeTextLowercase{\documentType}.	 Use the following checklist questions to help you in crafting your abstract.	

\begin{itemize}
	\item[$\square$] Did you briefly state what you intend to do?  
	\item[$\square$] Did you concisely discuss the problem statement?
	\item[$\square$] Did you tersely mention the objectives in general terms? 
	\item[$\square$] Did you succinctly describe the methodology for the target audience?
	\item[$\square$] Did you strongly describe your significant results and your conclusions?
\end{itemize}
\end{comment}
 Driver fatigue detection is very helpful in preventing vehicular accidents. These kinds of accidents happen all throughout the day especially in countries wherein public transport systems are under development and improvement. The two visual cues that will be analyzed carefully and integrated together in this study in order to accurately detect driver fatigue are the eye state and head pose. Eye tracking is a process of detecting, observing, and analyzing eye movements and its behavior; It utilizes an image capturing system that makes use of a developed algorithm for pupil detection and gaze estimation.  The images will be captured using an RGB-D sensor which can work with a microcontroller. Nowadays, this technology is becoming ubiquitous due to its growing number of applications. One of which is being able to analyze driver fatigue and the possibility of preventing road accidents. A carefully designed prototype that detects several levels of drowsiness will be mounted on strategic locations that is optimal for data collection and analysis. The levels of drowsiness will be characterized by the time the driver’s eyelids are closed, number of blinks, and head poses. Moreover, the said levels are based on psychological and scientific researches. In addition to eye tracking, analysis of head pose will also be included in order to utilize the depth values from the RGB-D sensor. The head pose refers to how tilted the head is to a specific direction. Thereafter, through head pose detection, it will also serve as a processing step for eye tracking. Having two visual cues, it will significantly help in reducing the number of false alarms caused by errors in the data gathered.



